\documentclass[a4paper]{elsarticle}
% vim: tw=0 wm=0

\setcounter{tocdepth}{3}
\usepackage{amssymb}
\usepackage{amsmath}
\usepackage{multirow}
\usepackage{longtable}
\usepackage{comment}
\usepackage{placeins}
\usepackage{mathtools}
\usepackage{algorithm}
\usepackage{algpseudocode}
\usepackage{enumitem}
\usepackage[utf8]{inputenc}
 \usepackage{booktabs}
\usepackage{array}
\usepackage[pdfencoding=auto,psdextra]{hyperref}
\usepackage{booktabs}
\usepackage{bookmark}% faster updated bookmarks
\usepackage{hypcap} % fix the links
\evensidemargin\oddsidemargin
\usepackage{graphicx}
\pagestyle{plain}
\usepackage{xcolor}
\newcommand\ToDo[1]{\textcolor{red}{#1}}
 
\usepackage{xspace}
\usepackage{color}
\usepackage{epsfig}
\usepackage{caption}
\usepackage{subcaption}
\usepackage{mathrsfs}
\usepackage{amssymb}
\usepackage{amsmath}
\usepackage{amsthm}
%\usepackage{changes}
  
\usepackage{blindtext}

\usepackage{numprint}
\npdecimalsign{.}
\npthousandsep{}

\usepackage[draft,nomargin,inline]{fixme}
\fxsetface{inline}{\itshape}
\fxsetface{env}{\itshape}
%\fxuselayouts{margin}
%\fxuselayouts{inline}
\fxusetheme{color}

\usepackage{url}
\newcommand{\keywords}[1]{\par\aDSvspace\baselineskip
	\noindent\keywordname\enspace\ignorespaces#1}

\usepackage{tikz}
\usetikzlibrary{positioning}
\definecolor{canaryyellow}{rgb}{1.0, 0.94, 0.0}
\definecolor{brightgreen}{rgb}{0.4, 1.0, 0.0}
\definecolor{jazzberryjam}{rgb}{0.65, 0.04, 0.37}

%defining of command

\newcommand\floor[1]{\lfloor#1\rfloor}
\newcommand\ceil[1]{\lceil#1\rceil}
\newcommand\str[1]{\texttt{#1}}
\newcommand\pL[1][]{\ensuremath{p^{\mathrm{L}#1}}}
\newcommand\pR[1][]{\ensuremath{p^{\mathrm{R}#1}}}
\newcommand\qL{\ensuremath{q^\mathrm{L}}}
\newcommand\qR{\ensuremath{q^\mathrm{R}}}
\newcommand\pLH{\ensuremath{\hat{p}^\mathrm{L}}}
\newcommand\pRH{\ensuremath{\hat{p}^\mathrm{R}}}
\newcommand{\Vext}{\ensuremath{V_\mathrm{{ext}}}}
\newcommand\UB{\ensuremath{\mathrm{UB}}}
\newcommand\Sigmand{\ensuremath{\Sigma^\mathrm{nd}}}
\newcommand{\mdmwnpp}{MDMWNPP\xspace}

\renewcommand{\labelenumii}{\theenumii}
\renewcommand{\theenumii}{\theenumi.\arabic{enumii}.}
\setlength{\leftmarginii}{1.8ex}
\raggedbottom
\algnewcommand\algorithmicforeach{\textbf{for each}}
\algdef{S}[FOR]{ForEach}[1]{\algorithmicforeach\ #1\ \algorithmicdo}

% scaling factor for tables
\newcommand\tabscale{0.8}
\newtheorem{Theorem}{Theorem}
\newtheorem{Lemma}{Lemma}

\begin{document}
	
	%\setlength{\parindent}{0pt}  % disallow indentations
	%\numberwithin{table}{1}
	%\mainmatter  % start of an individual contribution
	
	% first the title is needed
	\title{RILS-ROLS: Robust Symbolic Regression via Iterated Local Search and Ordinary Least Squares}
	
\author[1]{Aleksandar Kartelj}
\author[2]{Marko Djukanovi\'c}
	\address[1]{$kartelj@matf.bg.ac.rs$, \\  Faculty of Mathematics, University of Belgrade, Serbia}
    \address[2]{$ marko.djukanovic@pmf.unibl.org$,\\   Faculty of Natural Sciences and Mathematics, University of Banja Luka, Bosnia and Herzegovina}
	\begin{abstract}
		
	\end{abstract}
	\maketitle
	
	
\section{Introduction}\label{sec:introduction}
	
	The problem of symbolic regression (SR)~\cite{billard2002symbolic} has attracted many researchers over the last decade to study it intensively. SR can be seen as a generalization of the well known  concept of linear regression, i.e., polynomial regression~\cite{stimson1978interpreting}. All regression models in principle have the same task: given a set of a $n$-dimensional input data and the output data, the aim is to find a  function consisting of $n$ (input) variables that fits to the output data w.r.t. some in advance known measure. When aiming a model to be a linear combination of input variables, it represents the problem of linear regression. However, when there are some nonlinear relations between variables, which are relevant, linear regression models are not enough. This is the point where symbolic regression comes into the play. Unlike linear regression, it allows the search over the space of all possible mathematical formulas in order to find the best fitting ones able to  predict the output variable from the input variables. The base of constructing the explicit formula are the basis operations like addition and multiplication, as well as polynomial, trigonometric, exponential, and other elementary functions.  \fxnote{Marko: proceed}
	%Application: https://towardsdatascience.com/real-world-applications-of-symbolic-regression-2025d17b88ef
	 
	
\subsection{Contributions of this work}

The main contributions of this work can be summarized as follows:

\begin{enumerate}
	\item The proposed method outperforms all state-of-the-art methods for ground-truth problems from literature, considered in the SRBench benchmark. 
	
	\item It shows high robustness, which is proved by testing methods under different levels of Gaussian white noise. 
	
	\item The method is very efficient, taking in average around xyz seconds to reach exact solution -- when exact solution is found. 
	
	\item The new set of unbiased instances is introduced -- randomly generated formula of various sizes. All the other methods were tested against this new test-bed. The proposed RILS-ROLS method showed to be dominant here as well. 
\end{enumerate}


\section{Problem definition and search space}
\label{sec:search-space}

\section{Literature review}\label{sec:lit-rev}


\section{Proposed iterated local search method}\label{sec:rils}
   

\section{Experimental evaluation}\label{sec:experiments}

\subsection{Parameter tuning}
SAMO JEDAN PARAMETAR!!!

\subsection{Comparison with other methods}

\subsection{Statistical evaluation}

\section{Conclusions and future work}\label{sec:conclusions}
  
 \newpage
 \appendix
 
 \section{Complete results for a single random seed}\label{sec:appendix-1}
 
 \section{Overview of RILS-ROLS python package}\label{sec:appendix-2}
 
 \section{Analysis within SRBench}
 
 
\newpage
%\section*{References}
\bibliographystyle{abbrv}	
\bibliography{bib}	

	
\end{document}
