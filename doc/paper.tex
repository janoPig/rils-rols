\documentclass[a4paper,12pt]{elsarticle}
% vim: tw=0 wm=0

\setcounter{tocdepth}{3}
\usepackage{amssymb}
\usepackage{amsmath}
\usepackage{multirow}
\usepackage{longtable}
\usepackage{comment}
\usepackage{placeins}
\usepackage{mathtools}
\usepackage{algorithm}
\usepackage{algpseudocode}
\usepackage{enumitem}
\usepackage[utf8]{inputenc}
 \usepackage{booktabs}
\usepackage{array}
\usepackage[pdfencoding=auto,psdextra]{hyperref}
\usepackage{booktabs}
\usepackage{bookmark}% faster updated bookmarks
\usepackage{hypcap} % fix the links
\evensidemargin\oddsidemargin
\usepackage{graphicx}
\pagestyle{plain}
\usepackage{xcolor}
\newcommand\ToDo[1]{\textcolor{red}{#1}}
 \usepackage{tabularx}
\usepackage{xspace}
\usepackage{color}
\usepackage{epsfig}
\usepackage{caption}
\usepackage{subcaption}
\usepackage{mathrsfs}
\usepackage{amssymb}
\usepackage{amsmath}
\usepackage{amsthm}
%\usepackage{changes}
\usepackage{tikz}
\usepackage{fullpage}
\usepackage{calc}
\usetikzlibrary{positioning,shadows,arrows,trees,shapes,fit}
\usepackage{blindtext}
\usepackage{pgfplots}
\pgfplotsset{compat=1.16}

\usepackage{numprint}
\npdecimalsign{.}
\npthousandsep{}

\usepackage[draft,nomargin,inline]{fixme}
\fxsetface{inline}{\itshape}
\fxsetface{env}{\itshape}
%\fxuselayouts{margin}
%\fxuselayouts{inline}
\fxusetheme{color}

\usepackage{url}
\newcommand{\keywords}[1]{\par\aDSvspace\baselineskip
	\noindent\keywordname\enspace\ignorespaces#1}

\usepackage{tikz}
\usetikzlibrary{positioning}
\definecolor{canaryyellow}{rgb}{1.0, 0.94, 0.0}
\definecolor{brightgreen}{rgb}{0.4, 1.0, 0.0}
\definecolor{jazzberryjam}{rgb}{0.65, 0.04, 0.37}

%defining of command

\newcommand\floor[1]{\lfloor#1\rfloor}
\newcommand\ceil[1]{\lceil#1\rceil}
\newcommand\str[1]{\texttt{#1}}
\newcommand\pL[1][]{\ensuremath{p^{\mathrm{L}#1}}}
\newcommand\pR[1][]{\ensuremath{p^{\mathrm{R}#1}}}
\newcommand\qL{\ensuremath{q^\mathrm{L}}}
\newcommand\qR{\ensuremath{q^\mathrm{R}}}
\newcommand\pLH{\ensuremath{\hat{p}^\mathrm{L}}}
\newcommand\pRH{\ensuremath{\hat{p}^\mathrm{R}}}
\newcommand{\Vext}{\ensuremath{V_\mathrm{{ext}}}}
\newcommand\UB{\ensuremath{\mathrm{UB}}}
\newcommand\Sigmand{\ensuremath{\Sigma^\mathrm{nd}}}
\newcommand{\mdmwnpp}{MDMWNPP\xspace}

\renewcommand{\labelenumii}{\theenumii}
\renewcommand{\theenumii}{\theenumi.\arabic{enumii}.}
\setlength{\leftmarginii}{1.8ex}
\raggedbottom
\algnewcommand\algorithmicforeach{\textbf{for each}}
\algdef{S}[FOR]{ForEach}[1]{\algorithmicforeach\ #1\ \algorithmicdo}

% scaling factor for tables
\newcommand\tabscale{0.8}
\newtheorem{definition}{Definition}
\newtheorem{Lemma}{Lemma}

\begin{document}
	
	%\setlength{\parindent}{0pt}  % disallow indentations
	%\numberwithin{table}{1}
	%\mainmatter  % start of an individual contribution
	
	% first the title is needed
	\title{RILS-ROLS: Robust Symbolic Regression via Iterated Local Search and Ordinary Least Squares}
	
\author[1]{Aleksandar Kartelj}
\author[2]{Marko Djukanovi\'c}
	\address[1]{$kartelj@matf.bg.ac.rs$, \\  Faculty of Mathematics, University of Belgrade, Serbia}
    \address[2]{$ marko.djukanovic@pmf.unibl.org$,\\   Faculty of Natural Sciences and Mathematics, University of Banja Luka, Bosnia and Herzegovina}
	\begin{abstract}
		  \fxnote{TODO}
	\end{abstract}
	\maketitle
	
	
\section{Introduction}\label{sec:introduction}
	
	The problem of symbolic regression (SR)~\cite{billard2002symbolic} has attracted many researchers over the last decade to study it intensively. SR can be seen as a generalization of the well known  concept of linear regression, i.e., polynomial regression~\cite{stimson1978interpreting}. All regression models in principle have the same task: given a set of a $n$-dimensional input data and the output data, the aim is to find a  mathematical expression (function) consisting of $n$ (input) variables that fits to the output data w.r.t. some in advance known measure.  This computationally intensive task is in general  provenly NP--hard~\cite{virgolin2022symbolic}. When aiming a model to be a linear combination of input variables, it represents the problem of linear regression. However, when there are some nonlinear relations between variables, which are relevant, linear regression models are not enough. This is the point where symbolic regression comes into the play. Unlike linear regression, it allows the search over the space of all possible mathematical formulas in order to find the best fitting ones able to  predict the output variable from the input variables. The base of constructing the explicit formula are the basis operations like addition and multiplication, as well as polynomial, trigonometric, exponential, and other elementary functions.  
	%Application: https://towardsdatascience.com/real-world-applications-of-symbolic-regression-2025d17b88ef
	
	
	 Concrete examples of application of SR include having a less black--box tool, as it may say much more how SR model achieve predictions, that is the coefficients and functions that build the model may indicate on larger importance of some variables over the others. Moreover, in that way we may grasp why the variables are related in the obtained way. As an example, appearance of an exponential function may associate  to  some physical phenomenon such as intensity of radiation or acceleration over time, see~\cite{udrescu2020ai}. Additionally, the SR models, if they are correct, may have    posses  larger extrapolation power than other mathematical models, and especially than neural network--based methods. SR models may maximize obtaining physically more reasonable equations that could serve as insight towards  establishing  physical theory behind as the final goal.  Practical applications of SR in chemical and biological sciences are listed in~\cite{weng2020simple}. In particular,  the discovery of a series of new oxide perovskite catalysts with improved activities is presented there. Applications of SR to discovering physical laws from distorted video is studied in~\cite{udrescu2021symbolic} by means of a unsupervised learning method. Revealing complex ecological dynamics by SR is presented in~\cite{chen2019revealing}, tackled by  a machine learning method. Application of SR to model the mutations effects on protein  stability, in the domain of fundamental and applied biology,  is shown in~\cite{louis2021reviewing}. One of the recent work studies of  auto-discovering conserved quantities using trajectory data from unknown dynamical systems, where SR is solved by a machine learning algorithm, see~\cite{liu2021machine}. Last but not least, we mention the paper~\cite{liang2019phillips} discovers  the application of  SR to model analytic representations of the exciton binding energy. The use of solving the SR in material science is described in~\cite{wang2019symbolic,wang2022symbolic,burlacu2022symbolic,kabliman2021application}. Yet another application to wind speed forecasting is given in~\cite{abdellaoui2021symbolic}. 
	 
	 
	 
	 There are many different ways to tackle the SR by achieving the right analytic expressions, most of them are related to machine learning techniques or to the technique of genetic programming (GP) and various approximate methods, such as meta-heuristics. Among the first GP methods to tackle SR is the one of Raidl~\cite{raidl1998hybrid} based on a hybrid variant of genetic programming. A differential evolution algorithm was proposed by Cerny et al.~\cite{cerny2008using}. Age-fitness Pareto Optimization approach is proposed by Smidt and Lipson~\cite{schmidt2010age}.  Application of artificial bee colony programming to solve SR is proposed by Karaboga et al.~\cite{karaboga2012artificial}. 
	 Influence of local--based heuristics on solving SR is reported by Commenda in his Ph.D. thesis~\cite{kommenda2018local}. A  GP-based approach, the gene-pool optimal mixing evolutionary algorithm (GOMEA) is studied by Virgolin et al.~\cite{virgolin2021improving}.  Another evolutionary algorithm, the interaction-transformation EA (Itea) has been proposed by de Franca et al.~\cite{de2021interaction}. Simulated annealing to solve SR is proposed by Kantor~\cite{kantor2021simulated}. A Variable neighborhood programming approach to solve SR  is proposed by Elleurich et al.~\cite{elleuch2020variable}; this technique is initially proposed in~\cite{elleuch2016variable}. 
	 Kommenda et al.~\cite{kommenda2020parameter} proposed a method called OPERON algorithm, which uses nonlinear least squares for parameter identification of SR models further integrated into a local search mechanism in
	 tree-based GP. The C++ implementation of OPERON is discussed in~\cite{burlacu2020operon}. The method that uses Taylor polynomial to approximate the symbolic equation that fits the dataset, called Taylor genetic programming, is proposed in~\cite{he2022taylor}. A GP approach that uses the idea of semantic back-propagation (Sbp-Gp) is proposed in~\cite{virgolin2019linear}.   Empirical analysis of variance between many GP--based methods for SR is discussed in~\cite{kammerer2021empirical}. It is also worth to mention the GP--based Eurequa commercial solver~\cite{schmidt2009distilling, schmidt2011machine} that uses age-fitness pareto optimization  with co-evolved fitness estimation. This solver is nowadays accepted for the gold standard of symbolic regression.   
	 
	 
	 Machine learning--based method   based on Bayesian symbolic regression (Bsr) is proposed in~\cite{jin2019bayesian},  belongs to the family of Markov Chain Monte Carlo algorithms. Deep Symbolic Regression (Dsr), a RNN approach, is proposed in~\cite{petersen2019deep} utilizing the policy gradient search. This mechanism of search is further investigated in~\cite{landajuela2021improving}. A fast neural network approach, called OccamNet,  is proposed in~\cite{costa2020fast}.  A deep reinforcement learning approach enhanced with genetic programming is given in~\cite{mundhenk2021symbolic}. 
	
	Powerful hybrid techniques to solve SR are also well studied; among them, we emphasize the Eplex solver from~\cite{la2019probabilistic,la2016epsilon} and AI Feynman algorithm from~\cite{udrescu2020ai}, a physics-inspired divide-and-conquer method that  combines neural network
	fitting. The later one is one of the most efficient method for physically inspired models. We also mention the Fast Function Extraction (Ffx) algorithm developed by McConaghy~\cite{mcconaghy2011ffx},  a non-evolutionary method in combination with a machine learning technique---path-wise regularized learning--- which   quickly prune a huge set of candidate basis functions down to compact models.
	
	A short chronological overview of the most important  literature methods to solve SR are given in Table~\ref{tab:gp-based}. 
	
	
	 \begin{table}[!ht]
		\centering
		\scalebox{0.7}{
		\begin{tabularx}{550pt}{l  l  X}  
			Algorithm          &   Paper/year &   Short details   \\ \hline
			\textsc{Gp}        &    \cite{koza1994genetic} (1994)    & Application of GP to SR \\ 
	        \textsc{Hybrid-Gp}  &   \cite{raidl1998hybrid} (1998)   & GP--based method; solutions locally optimizes by method of least squares to find optimum coefficients for top-level terms \\ 
	        \textsc{Dfe}        & \cite{cerny2008using} (2008)   & Differential evolution algorithm \\
	        \textsc{Apf-Fe}                & \cite{schmidt2009distilling, schmidt2011machine} (2009, 2011) & Age-fitness pareto optimization approach using co-evolved fitness estimation   \\
        	APF            &      \cite{schmidt2010age} (2010)        &  Age-fitness pareto optimization approach                        \\
        	\textsc{Ffx}  & \cite{mcconaghy2011ffx} (2011)  & The Fast Function Extraction algorithm --  non-evolutionary technique based on a machine learning technique called path-wise regularized learning \\
             \textsc{Abcp} &  \cite{karaboga2012artificial} (2012)  & Artificial bee colony programming  approach \\
			\textsc{Eplex}          &      \cite{la2016epsilon} (2016)         &   A parent selection method called $\epsilon$--lexicase selection  \\ 
			\textsc{Mrgp} & \cite{arnaldo2014multiple} (2014) & It decouples and
			linearly combines a program's subexpressions via multiple regression on the target variable  \\
	       Local optimization NLS    & \cite{kommenda2018local} (2018)   & Constants Optimization in GP by Nonlinear Least Square \\
	       \textsc{Feat} & \cite{la2018learning} (2018) & Features are represented as networks of multi-type expression trees  comprised of activation functions; Differentiable features are trained via gradient descent \\
	       \textsc{Sbp-Gp} &  \cite{virgolin2019linear} (2019)  & The idea of semantic back-propagation utilized in GP \\
	       \textsc{Bsr} & \cite{jin2019bayesian} (2019) & ML--based approach; Bayesian symbolic regression  \\ 
	       \textsc{Dsr}  & \cite{petersen2019deep} (2019) & Deep Symbolic Regression based on a RNN approach further utilizing the policy gradient search \\
	       \textsc{Operon} & \cite{kommenda2020parameter} (2020) &  Utilizing nonlinear least squares for parameter identification of SR models with LS \\
	       \textsc{Vnp}    & \cite{elleuch2020variable} (2020) & VNS based GP approach \\
	       \textsc{OccamNet} & \cite{costa2020fast} (2020) &   A fast neural network approach; the model defines a probability distribution over a non-differentiable function space; it samples functions and updates the weights with back-propagation  based on cross-entropy matching in an EA strategy	 \\
	       
	       \textsc{AI-Feynman} & \cite{udrescu2020ai} (2020) & A physics-inspired divide-and-conquer method; it  combines neural network fitting \\
	       \textsc{Gomea}  & \cite{virgolin2021improving} (2021)    & A model-based
	       EA framework called Gene-pool optimal mixing evolutionary algorithm \\
	       \textsc{Itea} & \cite{de2021interaction} (2021)   & EA based approach called the interaction-transformation EA   \\
	       \textsc{Sa} & \cite{kantor2021simulated} (2021) &  Simulated annealing approach \\
	       
         \textsc{Drla} & \cite{mundhenk2021symbolic} (2021)  &    A deep reinforcement learning approach enhanced with genetic programming \\
          \textsc{Taylor-Gp} &  \cite{he2022taylor} (2022)  &  Taylor polynomials approximations  \\ \hline
         

	\end{tabularx} }
		\caption{Overview of methods to solve SR.}
		\label{tab:gp-based}
	\end{table}
 
  
	The progress in the SR field suffered from a lack of uniform, robust, and transparent
	benchmarking standards over the last two decade. Recently, La Cava et al.~\cite{la2021contemporary} proposed an
	open-source, reproducible benchmarking platform for SR, called SRBench. The authors extended  PMLB~\cite{olson2017pmlb},  a repository of standardized regression, by 130 SR datasets for which model forms are known; see more in the aforementioned paper. In the extensive experimental evaluation  where 14
	symbolic regression methods and 7 machine learning methods are compared on the  set of 252 diverse
    regression problems. The general conclusions derived from there are: ($i$) when comes to real-world problem, the best performing algorithms are those which combine genetic algorithm with parameter estimation; ($ii$) when come to the instances with the presence of noise, the GP--based and deep learning techniques are performing similarly. 
    
    In this work we present a novel approach to solve SR, which combines the popular iterated local search meta-heuristic (ILS)~\cite{lourencco2003iterated,lourencco2019iterated} with ordinary least squares method (OLS)~\cite{leng2007ordinary}. Briefly, ILS handles combinatorial (discrete) aspects of search space, while OLS helps in the process of coefficient determination, so it handles continuous parts of the search space. As will be shown later, the proposed method seems to be very robust w.r.t. introduced noise, therefore we called the method RILS-ROLS (with letters R corresponding to regression and robust).  \fxnote{TODO: maybe expand the core of the algorithm... }
 
The main contributions of this work may be summarized as follows:

\begin{enumerate}
	\item The proposed method outperforms all state-of-the-art methods for ground-truth problems from literature, considered in the SRBench benchmark. 
	
	\item It shows high robustness, which is proved by comparison to other algorithms under different levels of Gaussian white noise. 
	
	\item The method is very efficient, taking in average around xyz seconds to reach exact solution -- when exact solution is found. 
	
	\item The new set of unbiased instances is also introduced -- randomly generated formula of various sizes and number of input variables. This set was employed to understand the influence of formula size, number of input variables and noise on the solving difficulty. 
\end{enumerate}


\section{Problem definition and search space}
\label{sec:search-space}
In this section we formally define the SR problem, followed by defining the search space framework of the problem. 

\begin{definition}
  Given is a dataset $D = \{(x_i, y_i)\}_{i=1}^n$, where $\textbf{x} \in \mathbb{R}^d$ represents input variables (features), and $y$ target variable. Suppose that there exists an analytical model of the form $f(\textbf{x})= g^*(\textbf{x}, \theta^*) + \epsilon $ that is a generator of all observations from $D$.  
   The goal of SR is to learn a mapping $\tilde{f}(\textbf{x})=  \tilde{g}(\textbf{x}, \tilde{\theta})  \colon \mathbb{R}^d \mapsto \mathbb{R}$  estimated by searching through the space of (mathematical) expressions  $\tilde{g}$ and parameters $\tilde{\theta}$ where  $\epsilon$ is the presented  white noise using given input data. 
  
\end{definition}

Koza~\cite{koza1994genetic} introduced the problem of SR as a specific application of genetic programming. GP deals with the object called programs, which need to be optimized. In particular, the programs are represented by syntax trees consisting of functions/operations over input features and constants; as an example of a function represented by a syntax tree, see Figure~\ref{fig:syntax-tree-example} In essence, syntax trees are elements of the search space of SR. That is, each sample model $\tilde{f}$ may be seen as a point in the search space, represented by respective syntax tree. How accurate is this mapping, we compute in the basis of the historical data $D$ and the chosen error   measure (such as $MSE$, or $R^2$). Interestingly, the search space of SR consists of its discrete and continuous part. More precisely, the model we are seeking for might be obtained as a composition of some functions from a (finite) set of mathematical functions. It is common to use the set of the following elementary mathematical functions in the experiments: $\sqrt{x}, x^2 $, $\sin$, $\cos$, $\log$, $\exp$, $\arcsin$, $\arccos$, and $a^x$. It is allowed to link functions will all four operators: $+$, $-$, $\times$, and $/$. After fixing functions that build models, the set of optimal coefficients that  fits to this model must be provided. For the most physical models, these coefficients may receive any real value, that is they are continuous in its nature. 


\fxnote{TODO: needs for extension...}

\begin{figure}[!ht]
	\centering
\begin{tikzpicture}
	[font=\small, edge from parent, 
	every node/.style={top color=white, bottom color=blue!25, 
		rectangle,rounded corners, minimum size=6mm, draw=blue!75,
		very thick, drop shadow, align=center},
	edge from parent/.style={draw=blue!50,thick},
	level 1/.style={sibling distance=3cm},
	level 2/.style={sibling distance=1.2cm}, 
	level 3/.style={sibling distance=1cm}, 
	level distance=2cm,
	]
	\node (A) {$\times$} 
	child { node (B) {$\sin$}
		%child { node {x} 
			%edge from parent node[left=.5em,draw=none] {} }
		child { node {$x$}}
	}
	child {node (C) {$\times$}
		%child { node {x}
		%	child { node {C1a}}
		%}
		child { node {$x$}}
		child { node {$/$}
			child { node {$x$}}
			child { node {$y$} %edge from parent node[right=.5em,draw=none] {$\frac{a}{b}$}
			}
		}
	};
%	child { node {+} 
	%	child { node {2}}
	%	child { node {$x$}}
	%};
	
\end{tikzpicture}

\caption{Syntax tree representation for the expression $\sin x \times   ( x \times  x / y  )  = \frac{x^2 \sin x }{y}$}
\label{fig:syntax-tree-example}
\end{figure}

%\section{Literature review}\label{sec:lit-rev}


\section{The proposed method}\label{sec:rils}
   Before we give details on our RILS-ROLS method for solving SR, we will explain the basics of two main techniques incorporated into our algorithm. These are ILS meta-heuristic and OLS methods. 
   
  \subsection{Iterated local search}
    Iterated local search (ILS)~\cite{lourencco2003iterated} is an efficient meta-heuristics, whose essential idea is given as follow. Iteratively generate a sequence of solutions produced by the (embedded) heuristic, such as local search or randomized greedy heuristics. When the search gets \emph{stuck} in local optimum, a \emph{perturbation} is performed -- this step is usually non-deterministic. This simple idea appeared in the literature by Baxter~\cite{baxter1981local} in early 1980s, and, since then, has been  re-invented by many researches under different names; some of the following names were used: iterated descent search~\cite{baum1998iterated}, large-step
    Markov chains~\cite{martin1991large}, chained local optimization~\cite{martin1996combining}, or, in some cases, combinations of these~\cite{applegate2003chained}. 
  
   The most popular version of ILS is shown in Algorithm~\ref{alg:ils}. Note that we use this version of ILS as the backbone of our RILS-ROLS algorithm.
   
   \begin{algorithm}
  		\begin{algorithmic}[1] 
  			 \State \textbf{Input}: problem instance
  	 		\State \textbf{Output}: an approximate (feasible) solution 
  	 		\State $s \gets$ \texttt{Initialize}$()$
  	 		\State  $s \gets$ \texttt{LocalSearch}($s$)
   	 		\While{\emph{stopping criteria is not met}}
   	    		 \State  $s' \gets$ \texttt{Perturbation}($s$)
   	    		 \State  $s' \gets$ \texttt{LocalSearch}($s'$)
   	    		 \State  $ s \gets$ \texttt{AcceptanceCriterion}($s, s'$)
   	 		\EndWhile
   	 		\State \Return $s$
   		\end{algorithmic}
    	\caption{An ILS framework from   literature.}
    	   	     \label{alg:ils}
    \end{algorithm}  
   
An initial solution may be generated randomly or by using a greedy heuristic, which is afterwards, improved by local search. At each iteration, ILS applies three steps. First, the current incumbent solution is perturbed -- it is partially randomized. Next, the perturbed solution is improved with LS procedure. Thirdly, the LS outcome is thereafter compared to the incumbent solution and the search us possibly moved to the new solution, with accordance to the chosen acceptance criterion. Sometimes, ILS incorporates a mechanism of tabu list, which prevents the search getting back to already visited solutions. 
   
 \subsection{Ordinary least square method}\label{sec:ols}
   Ordinary least square method (OLS) is a linear regression technique. It is based on applying the least-square method to minimize the square residual (error) sum  between actual and predicted values (given by the model). More precisely, given data $({X}=(X_1, \ldots, X_d)^T, y)$, the task is to determine linear mapping $\tilde{y} = k \textbf{x} + b$, that is coefficients (line slope) $k = (k_1, \ldots, k_d)$ and $b$ (intercept), so that $ \sum_{i} |\tilde{y}_i - y_i|^2 $ is minimized. This sum is also know as the sum of squared error (SSE). There are many methods to minimize SSE. One of the analytical approaches is the calculus-oriented -- it takes into account   partial derivation w.r.t. $k$ of the cost function $ \sum_{i} |\tilde{y}_i - y_i|^2 $, getting 
   
   $$  \frac{\partial}{\partial k} \sum_{i} |\tilde{y}_i - y_i|^2 = \frac{\partial}{\partial k} \sum ( kX_i+b  - y_i)^2 =  \sum 2X_i(kX_i + b - y_i)  =  \sum -2X_i (y_i - \tilde{y}_i) . $$
   
   Taking into account   partial derivation w.r.t.  $b$, we have 
    $$  \frac{\partial}{\partial b} \sum_{i} (\tilde{y}_i - y_i)^2  = \frac{\partial}{\partial b} \sum ( kX_i+b  - y_i)^2 = \sum -2  (y_i - \tilde{y}_i). $$
    
    Thus, we get the system of equations: 
    \begin{align*}
    	 % &\sum -2X_i (\tilde{y}_i - y_i )  = 0 \\
    	  & -2X_1 (y_1 - \tilde{y}_1)  = 0 \\
    	  & \vdots \\
    	  & -2X_d (y_d - \tilde{y}_d)  = 0 \\
    	  &\sum -2  (y_i - \tilde{y}_i) = 0.
    \end{align*}
     %The above system can be transformed to 
     %\begin{align*}
     %	 & \sum y_i \cdot x_i   = b\sum x_i \cdot x_i  +  k \cdot \sum x_i \\
     %	 & \sum y_i  = k \sum x_i + b \cdot d. 
     %  \end{align*}
     By solving this system of $(d+1)$ equations, one can obtain vector of coefficients $k=(k_1, ..., k_d)$ and intercept $b$. %For example, in order to calculate coefficient $k_i$, the following system of equations is 
     \fxnote{TODO: Aca -- ovde je pisalo da je to sistem sa dve jednacine i dve nepoznate k i b -- medjutim, k je vektor. Ako mozes dopisi jos nesto, tipa koliko je podataka potrebno i slicno -- mozda veremenska slozenost.) }%https://towardsdatascience.com/understanding-the-ols-method-for-simple-linear-regression-e0a4e8f692cc
   \subsection{RILS-ROLS method}
   Now we will explain the proposed RILS-ROLS method in detail. The overall method scheme is given in Algorithm~\ref{alg:rilsrols}.  
   
    \begin{algorithm}
    \footnotesize
    \hspace*{\algorithmicindent} \textbf{Input}: input training dataset $D_{tr}$ \\
     \hspace*{\algorithmicindent} \textbf{Control parameters}: size penalty $penalty_{size}$, error tolerance $tolerance_{error}$ \\
    \hspace*{\algorithmicindent} \textbf{Output}: best symbolic formula solution $bs$
   	\begin{algorithmic}[1] 
   		\Procedure{RILS-ROLS}{$D_{tr}$}
   		\State $n_{tr} \gets |D_{tr}|$
   		\State $sample_{size} \gets $ \texttt{InitialSampleSize}($n_{tr}$)
   		\State $D_{tr}' \gets$ \texttt{Sample}($D_{tr}, sample_{size}$)
   		\State $s \gets NodeConstant(0)$ 
   		\State $s_{fit} \gets$ \texttt{Fitness}($s, D_{tr}'$)
   		\State $bs, bs_{fit} \gets s, s_{fit}$ \label{line:solSet}
   		\State $start_{tried}, perturbations_{tried} \gets \emptyset, \emptyset$
   		\While{\emph{stopping criteria is not met}}
   			\State $start_{tried} \gets start_{tried} \cup \{s\}$
   			\State $s_{perturbations} \gets $ \texttt{All1Perturbations}($s$) \label{line:sPert}
   			\State $s_{perturbations} \gets $ \texttt{FitOLS}($s_{perturbations}$, $D_{tr}'$)
   			\State $s_{perturbations} \gets $ \texttt{OrderByR2}($s_{perturbations}$) \label{line:orderR2}
   			\State $improved \gets false $
   			\For{$p \in s_{perturbations}$}
   				\If{$p \in perturbations_{tried}$}
   					\State \textbf{continue}
   				\EndIf
  				\State $p \gets $ \texttt{Simplify}($p$) \label{line:simp}
  				\State $p \gets $ \texttt{LocalSearch}($p$, $D_{tr}'$) \label{line:ls}
  				\State $p_{fit} \gets$ \texttt{Fitness}($p$, $D_{tr}'$)
  				\If{$p_{fit} < bs_{fit}$} \label{line:avoid1}
  					\State $bs, bs_{fit}, improved \gets p, p_{fit}, true$ // new best solution
  					\State \textbf{break}
  				\EndIf \label{line:avoid2}
  				\State $perturbations_{tried} \gets perturbations_{tried} \cup \{p\}$
   			\EndFor
   			\If{improved}
   				\State $s \gets bs$ \label{line:improved}
   			\Else
   				\State $start_{candidates} \gets $ \texttt{All1Perturbations}($bs$)
   				\If{$start_{candidates} \setminus start_{tried} = \emptyset$} // all 1-perturbations around $bs$ tried
   					\State $s' \gets $ \texttt{RandomPick}($start_{candidates}$)  \label{line:pert21}
   					\State $start_{candidates} \gets $ \texttt{All1Perturbations}($s'$) \label{line:pert22}
   				\EndIf
   				\State $s \gets $ \texttt{RandomPick}($start_{candidates} \setminus start_{tried}$) \label{line:randPick}
   				\If{not improved for too many iterations}
   					\State $sample_{size} \gets$ \texttt{IncreaseSampleSize}($sample_{size}, n_{tr}$) \label{line:sampleAdj1}
   					\State $D_{tr}' \gets$ \texttt{Sample}($D_{tr}, sample_{size}$)\label{line:sampleAdj2}
   				\EndIf
   			\EndIf
   			\If{$R^2$ almost 1 and RMSE almost 0 w.r.t. $tolerance_{error}$}
   				\State \textbf{break} // early exit
   			\EndIf
   		\EndWhile
   		\State $bs \gets $ \texttt{Simplify}($bs$)
   		\State $bs \gets $ \texttt{RoundModelCoefficients}($bs$)
   		\State \Return $bs$
   		\EndProcedure
   	\end{algorithmic}
   	\caption{RILS-ROLS method.}
   	\label{alg:rilsrols}
   \end{algorithm}  

RILS-ROLS algorithms receives training dataset $D_{tr}$ as input. In addition, it has two control parameters: $penalty_{size}$ and $tolerance_{error}$. The first one quantifies the importance of solution expression complexity in the overall solution quality measure (more on this in Section~\ref{sec:fitness}). The second parameter is related to the expected noise level in data -- higher noise means that tolerance to errors should be higher.
 $D_{tr}$ can have very large number of records, so the first step of RILS-ROLS is to choose a random sample $D_{tr}' \subseteq D_{tr}$ of size $sample_{size}$. Initial the sample size is calculated as $\max(0.01 \cdot n_{tr}, 100)$. The size of sample is later dynamically adjusted through the algorithm's iterations -- when there are no solution improvements for some number of iterations, the sample size is doubled (lines~\ref{line:sampleAdj1}-\ref{line:sampleAdj2} of Algorithm~\ref{alg:rilsrols}).


As previously stated, solution is usually represented by means of a tree. We use simple solution initialization -- tree root node is set to zero constant. We interchangeably use two solution variables: ($i$) $s$ denotes starting (or working) solution and ($ii$) $bs$ stands for the best solution so far. Solution quality is measured by evaluating the fitness function (more about it in the subsequent Section~\ref{sec:fitness}). Before entering the main loop, the best solution \emph{bs} is set to the initial solution (line~\ref{line:solSet}).


Main loop iterates as long as none of termination criteria is met: ($i$) maximal running time has been reached; ($ii$) maximal number of fitness calculations has been made; ($iii$) best solution is sufficiently good, 
w.r.t. the $R^2$ and $RMSE$ of the best solution. More precisely, if $R^2$ is sufficiently close to 1 and, at the same time, $RMSE$ is sufficiently close to 0, the algorithm stops prematurely -- this significantly reduces the running times for the majority of tested instances.
The sufficiency is controlled with parameter $tolerance_{error}$ -- smaller value means that $R^2$ and $RMSE$ should be closer to 1 and 0, respectively. 


One of the first steps in the main loop is to generate perturbations near the starting solution $s$ (line~\ref{line:sPert}). 
As the name of this procedure (\texttt{All1Perturbations}) suggests, perturbation step is local -- meaning the closeness of starting solution $s$ and any of perturbations is 1 (we call it 1-perturbation sometimes). The precise way of generating perturbation is described separately, in Section~\ref{sec:pertGen}. 


Candidate perturbations are being improved by performing OLS coefficient fitting (procedure \texttt{FitOLS}). This means that coefficients in any of linear combinations of current solution are being set by means of ordinary least squares method, described in Section~\ref{sec:ols}.   
After this step, perturbations are usually better suited to given sample data $D_{tr}'$. 
Further, these perturbations are sorted w.r.t. $R^2$ metric in the descending order (line~\ref{line:orderR2}). 


Now, the algorithm enters the internal loop -- it iterates over the ordered perturbations and aims to find the one which improves the best solution $bs$. But, before comparing candidate perturbation solution $p$ with $bs$, $p$ is first simplified (line~\ref{line:simp}) after which the local search is performed (line~\ref{line:ls}).
Solution simplification is done in a symbolical fashion by popular Python library called SymPy~\cite{sympy}.
Local search tries to find local optima expressions close to the given $p$ -- explained in details in Section~\ref{sec:ls}.  
Finally, fitness function value of $p$ is compared to fitness function value of $bs$. If a new incumbent is found, $bs$ is updated correspondingly and internal loop that goes across ordered perturbations is immediately terminated (it works in a  \emph{first-improvement} strategy). Otherwise, the next perturbation is probed. 
Note that the probed perturbations are stored in a set denoted by $perturbations_{tried}$. The goal is to avoid checking the same perturbation multiple times (lines \ref{line:avoid1}-\ref{line:avoid2}), i.e. this structure serves as a kind of tabu list, initially proposed in the famous Tablu search algorithm, see~\cite{glover1998tabu}.  


If some of $s_{perturbations}$ around starting solution $s$ yielded an improvement, $bs$ becomes the starting solution $s$ in the next iteration of the main loop (line~\ref{line:improved}). 
Otherwise, it makes no sense to set starting solution to $bs$, as nothing changed -- the search is being  \emph{trapped} in a local optimum. In order to avoid this undesired situation, a randomness needs to be triggered. First, a set of local perturbations around $bs$ is generated ($start_{candidates}$) in the same manner as before (procedure \texttt{All1Perturbations}). If at least one of these was not previously used as a starting solution ($start_{tried}$), a single perturbation from the $start_{candidates} \setminus start_{tried}$ is randomly picked (line~\ref{line:randPick}). There is a insignificant chance that $start_{candidates} \setminus start_{tried} = \emptyset$. When that happens, the set of starting solution candidates is equal to perturbations of some randomly selected perturbation of $bs$ (lines \ref{line:pert21}-\ref{line:pert22}) -- which effectively means that the perturbations with distance 2 from $bs$ are used. Indeed, there is a minor chance that these 2-perturbations will have empty set difference with $start_{tried}$ set of starting solutions. Therefore, more expensive 3-perturbations are not considered in our algorithm. 


Before returning the final symbolical model, RILS-ROLS peforms the final symbolical simplification and rounding of model coefficients. The later is sensitive to control parameter $tolerance_{error}$ -- smaller value means information loss during rounding is smaller, i.e. higher number of significant digits is preserved. Note that confidence in the symbolical model is reduced when expected noise in data is higher -- the rule of thumb is: for higher expected noise $tolerance_{error}$ should be higher. 


\fxnote{Dati neku smislenu procjenu kompleksnosti algoritma, ako to nije pretesko. }
\subsubsection{Fitness function}\label{sec:fitness}

Symbolic regression objective is to determine the expression governing available data. It is also allowed to obtain some equivalent expression, keeping in mind that there are multiple ways to express some symbolical equation. Although logically sound and intuitive, this objective is not quantifiable during the solution search/training phase, simply because the goal expression is not known at that point as only target values for some of the input features are known. Thus, different numerical metrics are being used in literature to guide the symbolic regression search process. The most popular are the coefficient of determination, also known as $R^2$ and mean squared error (MSE) or root mean squared error (RMSE). The important aspect of the solution quality might also be the solution expression complexity. This follows the Occam's razor principle~\cite{costa2020fast}~ that simpler solution  is more likely to be a correct one. 
The search process of RILS-ROLS is guided by the non-linear combination of $R^2$, RMSE and solution expression size (complexity) presented in Equation~\ref{eq:fitness}. 

\begin{equation}
	\label{eq:fitness}
	fit(s) = (2-R^2(s)) \cdot (1+RMSE(s)) \cdot (1+penalty_{size} \cdot size(S))
\end{equation}

Since the presented fitness function needs to be minimized, the following conclusions may be drawn:
\begin{itemize}
	\item higher $R^2$ is preferred -- ideally, when $R^2(s)=1$, the effect of term $1+R^2(s)$ is neutralized; %since 1 is neutral for multiplication;
	\item lower RMSE is preferred, ideally, when $RMSE(s)=0$, the whole term $(1+RMSE(s))$ becomes 1;
	\item since $penalty_{size} > 0$, larger expressions tend to have higher fitness (which follows the Occam's razor principle); therefore, we award solutions that are more simple with its representation. 
\end{itemize}

The size of expression is calculated by simply counting all nodes in the expression tree -- this includes leaves (variables and constants) and internal nodes (operations). 

\subsubsection{Perturbations}\label{sec:pertGen}

Perturbations allow the algorithm to escape from local optima. As previously described, perturbations are performed in two occasions: ($i$) during the exhaustive examination of neighboring solutions around the starting solution, ($ii$) during selection of the next starting solution, a non-exhaustive case.  
In both cases, the same Algorithm~\ref{alg:pert} is used. 

\begin{algorithm}
	\hspace*{\algorithmicindent} \textbf{Input}: solution $s$ \\
	\hspace*{\algorithmicindent} \textbf{Output}: local perturbations (1-perturbations) of solution $s$ -- $s_{perturbations}$
	\begin{algorithmic}[1] 
		\Procedure{All1Perturbations}{$s$}
		\State $s_{perturbations} \gets \emptyset$ \label{line:pertInit}
		\State $s \gets$ \texttt{NormalizeConstants}($s$)
		\State $s \gets$ \texttt{Simplify}($s$)
		\State $s_{subtrees} \gets$ \texttt{SubTrees}($s$)
		\For{$n \in s_{subtrees}$} \label{line:forSSStart}
			\State $n_{perturbations} \gets$ \texttt{All1PerturbationsAroundNode}($s$, $n$)
			\State $s_{perturbations} \gets s_{perturbations} \cup n_{perturbations}$	
		\EndFor \label{line:forSSEnd}
		\State \Return $s_{perturbations}$
		\EndProcedure
	\end{algorithmic}
	\caption{Generation of all 1-perturbations of a given solution.}
	\label{alg:pert}
\end{algorithm}  

Initially, the set of perturbations $s_{perturbations}$ is empty (line~\ref{line:pertInit} of Algorithm~\ref{alg:pert}).


This is followed by constant normalization during which coefficients that enter multiplication, division, addition or subtraction are set to 1, while those entering the power function are rounded to integer, with exception of square root, which is kept intact. For example, for expression $3.3\cdot(x+45.1\cdot y^{3.2})\cdot 81\cdot x/\sqrt{y}$ the normalized version is $1\cdot (x+1\cdot y^3)\cdot 1\cdot x/\sqrt{y}$. The reason for performing normalization is reducing the search space of possible perturbations thus keeping them within a reasonable size. This reduction is reasonable, since normalization preserves the essential functional form. (Note that coefficients get tuned later during the OLS phase and local search.)


After performing the normalization process, the expression is simplified -- getting compact expression is more likely after normalization than before. The previous expression will take form $(x+y^3)\cdot x/\sqrt{y}$. The purpose of simplification is removing unnecessary coefficients, but in general it can also be non-trivial symbolic simplification. 


Perturbations are generated by making simple changes on the per-node level of $s$ expression tree.
Depending on the structure of the expression tree (expression does not have to have unique tree representation), the set of subtrees of the previous expression $(x+y^3)\cdot x/\sqrt{y}$ might be $\{(x+y^3)\cdot x/\sqrt{y}, (x+y^3), x/\sqrt{y}, x, y^3, \sqrt{y}, y\}$. This set of subtrees is obtained if left subtree of the whole expression is $(x+y^3)$ while the right one is $x/\sqrt{y}$. 
For each subtree $n$, the set of perturbations around $n$ is generated (lines \ref{line:forSSStart}-\ref{line:forSSEnd} of Algorithm~\ref{alg:pert}).   


Algorithm~\ref{alg:pertNode} shows how perturbations around given subtree (node) are generated.

\begin{algorithm}	
	\hspace*{\algorithmicindent} \textbf{Input}: solution $s$, node $n$\\
	\hspace*{\algorithmicindent} \textbf{Output}: 1-perturbations of solution $s$ around node $n$
	\begin{algorithmic}[1]
		\Procedure{All1PerturbationsAroundNode}{$s$, $n$}
		\State $n_{perturbations} \gets \emptyset$
		\If{$n = s$} \label{alg:alg4-case-1}
		\State $n_{perturbations} \gets n_{perturbations} \cup$ \texttt{NodeChanges}($n$)
		\EndIf
		\If{$n.arity \geq 1$}\label{alg:alg4-case-2}
		\State $n_{changes} \gets$ \texttt{NodeChanges}($n.left$)
		\For{$nc \in n_{changes}$}
		\State $new \gets$ \texttt{Replace}($s$, $n.left$, $nc$)
		\State $n_{perturbations} \gets n_{perturbations} \cup \{new\}$
		\EndFor
		\EndIf
		\If{$n.arity = 2$}\label{alg:alg4-case-3}
		\State $n_{changes} \gets$ \texttt{NodeChanges}($n.right$)
		\For{$nc \in n_{changes}$}
		\State $new \gets$ \texttt{Replace}($s$, $n.right$, $nc$)
		\State $n_{perturbations} \gets n_{perturbations} \cup \{new\}$
		\EndFor
		\EndIf
		\State \Return $n_{perturbations}$
		\EndProcedure
	\end{algorithmic}
	\caption{Generation of 1-perturbations of a given solution around given node.}
	\label{alg:pertNode}
\end{algorithm}

It can be seen that there are three possibly overlapping cases when performing perturbations on the per-node level. 

\emph{Case 1}. 
Observed node $n$ is the whole tree $s$ (see line ~\ref{alg:alg4-case-1} in Algorithm~\ref{alg:pertNode}). Following on   the previous exemplary expression tree, this means that multiplication node that connects $(x+y^3)$ and $x/\sqrt{y}$ is to be changed. For example, multiplication can be replaced by addition, which forms a 1-perturbation expression (tree) $(x+y^3)+ x/\sqrt{y}$. 


\emph{Case 2}. 
Node $n$ has arity of at least 1 (see line ~\ref{alg:alg4-case-2} in Algorithm~\ref{alg:pertNode}). This means that the left subtree exists, so the left subtree node is to be changed. 
For example, if $n=x+y^3$ the overall perturbation might be $(x/y^3)+x/\sqrt{y}$ (addition is replaced by division). 
Another example would be the case of unary operation, e.g. when $n=\sqrt{y}$. In that case, some of the possible perturbations could be $(x+y^3)\cdot x/\sqrt{\ln{y}}$ (application of logarithm to left subtree $y$) or $(x+y^3)\cdot x/\sqrt{x}$ (changing variable $y$ to $x$), etc.

\emph{Case 3}. 
Subtree $ss$ is binary operation, meaning the right subtree must exist (Line ~\ref{alg:alg4-case-3} in Algorithm~\ref{alg:pertNode}). 
The analogous idea is applied as in \emph{Case 2}. 

The algorithm allows the following set of carefully chosen per-node changes:

\begin{enumerate}
	\item Any node to any of its subtrees (excluding itself). For example, if $(x+y^3)$ is changed to $x$, the perturbation is $(x+y^3)\cdot x/\sqrt{y} \rightarrow x\cdot x/\sqrt{y}$. 
	\item Variable to another variable or constant to variable. For example,  $(x+y^3)\cdot x/\sqrt{y} \rightarrow (y+y^3)\cdot x/\sqrt{y}$.
	\item Variable to unary operation applied to that variable. For example,  $(x+y^3)\cdot x/\sqrt{y} \rightarrow (x+y^3)\cdot \ln{x}/\sqrt{y}$.
	\item Unary operation to another unary operation. For example,  $(x+y^3)\cdot x/\sqrt{y} \rightarrow (x+y^3)\cdot x/\sin{y}$.
	\item Binary operation to another binary operation. For example,  $(x+y^3)\cdot x/\sqrt{y} \rightarrow (x+y^3)\cdot (x + \sqrt{y})$. 
	\item Variable or constant enter the binary operation with arbitrary variable. For example,  $(x+y^3)\cdot x/\sqrt{y} \rightarrow (x+x/y^3)\cdot x/\sqrt{y}$. 
\end{enumerate} 
\fxnote{Nekako zbunjuje u Algoritmu 4, pojava NodeChanges)() i Replace(). Mislim, razumijem iz opisa sta one rade, ali nisam siguran da treba imenovati metode, ako ih barem u kontekstu opisa ne pomenemo direktno po nazivu. }


\fxnote{Dati kompleksnost velicine skupa perturbacija u odnosu na velicinu drveta i dopustene promjene}

\subsubsection{Local search}\label{sec:ls}

Perturbations are further improved by means of local search procedure (Algorithm~\ref{alg:ls}). 

\begin{algorithm}
	\hspace*{\algorithmicindent} \textbf{Input}: perturbation $p$, sample training dataset $D_{tr}'$ \\
	\hspace*{\algorithmicindent} \textbf{Output}: local optimum $bp$ in the vicinity of perturbation $p$
	\begin{algorithmic}[1] 
		\Procedure{LocalSearch}{$p$, $D_{tr}'$}
		\State $bp, bp_{fit} \gets p,\ $\texttt{Fitness}$(p,D_{tr}')$ 
		\State $improved \gets true$
		\While{$improved$}
			\State $improved \gets false$
			\State $bp_{candidates} \gets \emptyset$
			\State $bp_{subtrees} \gets $\texttt{SubTrees}($bp$)
			\For{$n \in bp_{subtrees}$}
				\State $n_{candidates} \gets $ \texttt{All1PerturbationsAroundNode}($bp$, $n$)
				\For{$new \in n_{candidates}$}
					\State $new \gets$ \texttt{FitOLS}($new$, $D_{tr}'$)
					\State $new_{fit} \gets$ \texttt{Fitness}($new$, $D_{tr}'$)
					\If{$new_{fit} < bp_{fit}$}
						\State $bp, bp_{fit} \gets new, new_{fit}$
						\State $improved \gets true$
					\EndIf
				\EndFor
			\EndFor
		\EndWhile
		\State \Return $bp$
		\EndProcedure
	\end{algorithmic}
	\caption{Local search procedure.}
	\label{alg:ls}
\end{algorithm}  

For a given perturbation $p$, local search systematically explores all 1-perturbations around $p$. It relies on the \emph{best-improvement} strategy, meaning that all 1-perturbations (for all subtrees) are considered. Before checking if the candidate solution ($new$) is better that the actual best $bp$, the OLS coefficient fitting (\texttt{FitOLS}) is performed. 
   
\section{Experimental evaluation}\label{sec:experiments}

Our \textsc{Rils}--\textsc{Rols} algorithm is implemented in Python version 3.x. All experiments are performed on a machine xx with xx GHz and a memory limit of xx GB in single-threaded mode. The maximum computation time allowed for each run was set to 1h minutes, i.e., 3600 seconds.
 
The following 15 algorithms are compared: \textsc{AI-Feynman}, \textsc{GOMEA}, \textsc{Afp-Fe}, \textsc{Itea}, \textsc{Afp}, \textsc{Dsr}, \textsc{Operon}, the \textsc{Gplearn} from python package \textsc{gplearn}~\cite{stephens2016genetic}, \textsc{Sbp-Gp}, \textsc{Eplex}, \textsc{Bsr}, \textsc{Feat}, \textsc{Ffx}, \textsc{Mrgp} and our \textsc{Rils}-\textsc{Rols} algorithm.


 
\subsection{Datasets and SRBench}

SRBench is an open-source benchmarking project which merge a large set of diverse benchmark datasets, contemporary SR methods as well as ML methods around a shared model evaluation and the environment for analysis, see~\cite{la2021contemporary}. It brought researchers an easy handled environment to evaluation, compare, and analyze their methods. Among all benchmark sets, we involve two real-world inspired benchmark sets into our computational evaluation: 
\begin{itemize}
	\item \textsc{Feynman} benchmark set, inspired from physics and formulas/models that correspond to various natural laws.  
	It consists of $10^5$  samples, each described in~\cite{udrescu2020ai}. Some exact models (equations) of problem instances from this benchmark are shown in Table~\ref{tab:Feynamn-Eq}.  
	
   \begin{table}
		\centering
		\begin{tabular}{l}   \hline
	       $x = \sqrt{x_1^2 + x_2^2 - 2 x_1 x_2 \cos(\theta_1 - \theta_2)}$ \\
	       $ \theta_1 = \arcsin(n \sin \theta_2)$ \\
	       $E =  \frac{m c^2 }{1 - \frac{v^2}{c^2}}$ \\
	       $\omega = \frac{1 + \frac{v}{c}}{ \sqrt{1 - \frac{v^2}{c^2}}} \omega_0$ \\ \hline
	       
	    \end{tabular}
		\caption{Some Feynman equations.}
		\label{tab:Feynamn-Eq}
	\end{table}


	\item \textsc{Strogatz} benchmark set from~\cite{la2016inference}. 
	Each dataset is represented by a 2-state system of first-order, ordinary differential equations. 
	The aim of each problem is to predict rate of change of the subsequent state w.r.t\  the current two states on which it depends. These equations describes various non-linear dynamics and exhibiting chaos.  The equations for some of the datasets of this benchmark are shown in Table~\ref{table:strogatz-ODEs}.
	
	%table, an instance -- example
	
	\begin{table}
		\centering
		\begin{tabular}{ll} \\ \hline
			Bacterial representation &   $x' = 20 - x - \frac{x \cdot y}{1 + 0.5 x^2 }$ \\ 
			                         &   $y' = 10 - \frac{x \cdot y}{1 + 0.5 x^2  }$ \\ \hline
			Shear Flow               &  $\theta' = \cot(\phi)\cos(\theta)$ \\
			                         &  $ \phi'  = ( \cos^2(\phi) + 0.1 \cdot \sin^2 (\phi)) \sin(\theta) $ \\ \hline
		\end{tabular}
	    \caption{Some Strogatz ODE problems.}
	    \label{table:strogatz-ODEs}
	\end{table}
	 

\end{itemize}
The above-mentioned benchmark sets may be biased as models which describe physical laws may have various symmetries, periodically w.r.t. some variables, internal separability on some variables etc. In that way, the search process and the search space can be biased and in-advance significantly reduced. Thus, some methods may get significant help with this regards, while some others may be in the subordinate situation. 
  In order to permit any kind of favoritism, we generate a set of randomly generated instances, called \textsc{Random}. These instances are generated according the following procedure. \fxnote{Opisati proceduru generisanja instanci + kombinaciju parametara za instance, itd.}


\subsection{Parameter tuning}
The \textsc{Rils}-\textsc{Rols} algorithms, apart of the most other methods for SR, employs just two parameters that need to be tuned, $penalty_{size}$ and $tolerance_{error}$. The first parameter is set to 0.001(?), according to the preliminary experimental evaluation. The second parameter is tuned monitoring the two errors the algorithm produced on the training set; we choose $tolerance_{error}=?$ for which a largest average $R^2$ and the smallest average RMSE error over all instances from the training set are produced. Note that the training set is chosen randomly with a cardinalty of xx instances. 

\fxnote{ drugi parametar manje vise zavistan od ocekivanog 
nivoa suma -- ne znam kako obrazloziti njegov tuning (mozda prema greski na trening skupu pa izabrati onaj za koji su R2 i RMSE bolji...). }

\subsection{Comparison with other methods on real--world benchark sets}
In this section, we evaluate the performance of \textsc{Rils-Rols} algorithm with 15 other competitors from the literature that solve the problem of Symbolic regression. The results are reported in terms of three numerical Tables~\ref{tab:comp_noise0}--\ref{tab:comp_noise001}, where two real-world benchmark sets, \textsc{Feynman} and \textsc{Storgatz} are considered, utilizing  different values of the noise levels: 0.0 (no noise), 0.001 (a low-level noise), and 0.01 (a high-level noise), respectively. Concerning the added noise, the White Gaussian noise was added to the target ($y$-axis values) as a fraction of the signal RMS value, also applied in~\cite{la2021contemporary}, i.e. for target noise level $\alpha$, we have
$$ y_{noise} = y + \epsilon, \epsilon \sim \mathcal{N}\left(0, \alpha \sqrt{\frac{1}{N} \sum {y_i^2}}\right). $$

Each table consist of three blocks. The first block reports the algorithm whose performances are reported. The second block reports, in two columns,  percentages of instances where the model which matches to the exact one has been found by respective algorithms on both, \textsc{Feynman} and \textsc{Strogatz}, benchmark sets. The third column of this block provides the overall summary percentage of success over all instances. The third block displays the results of the respective algorithm in the presence of $R^2$ error which can indicate on  accuracy, i.e. in our cases when the algorithm  at least reached the accuracy ($R^2$ score) of 0.999.  Note that the termination of our algorithm may not necessary be triggered by the error equal to 1. These results are of interest as it may point out the existence of an over-fitting. This block consists of three rows that indicate the percentage of instances where respective algorithm is able to reach the desired accuracy (fulfilling condition $R^2 \geq 0.999$ before a termination due exceeding the time limit) on the both real--world benchmark sets separately, and the overall percentage of success rate (over all considered instances) w.r.t. the same criterion of success.

%Note that when the algorithm terminates with $R^2=1$ and $RMSE=0$, the obtained model and the exact model are always match (but also checked manually). However, in case when the algorithms deliver an error $R^2 \geq 0.999$, but $R^2 \neq 1$, these cases are all  manually checked by comparing weather the produced model match the exact model. 

In short, the exact results, thus the number of instances (in percentages) for which the output of respective algorithm match to the exact model, are displayed in the second block of each of the tables. The numbers of instances (in percentages) for which each algorithms deliver a precise enough model (i.e. satisfying $R^2\geq 0.999$ upon its termination) are displayed in the third block of each of the tables. 

The following conclusions may be drawn from the numerical results: 

\begin{itemize}
	\item Concerning the exact results in case no noise is utilized in the instance problems, the best performing algorithm is our \textsc{Rils}--\textsc{Rols}, able to find the right model on 57.84\% problems instances of the benchmark set \textsc{Feynman}; the same algorithm was even more impressive in solving problem instances from benchmark set \textsc{Strogatz}, begin successful in 83.57\% cases. The second best performing  approach is \textsc{AI-Feynman}, successful in solving 55.78\% and just 27.14\% of all instances of the benchmark set \textsc{Feynman} and \textsc{Strogatz}, respectively. All other approaches are performing significantly worse, and none of them is able to solve more than 30\% instances of any of the two considered benchmark sets. 
	\item   Concerning the exact results when the target noise is presented with a level of 0.001, our \textsc{Rils}--\textsc{Rols} is still performing reasonably well, having found an exact model in more than 42\% cases considering the problems from the both benchmark sets. Interestingly, \textsc{AI-Feynman} is quite sensible w.r.t.\ the increase in the target noise level, as its performances rapidly drop down -- it is successful on just 2.81\% of the all problem instances. Better from \textsc{AI-Feynman} are in this case the \textsc{Dsr}, \textsc{Gplearn}, and \textsc{Afp-Fe} in this order of success, whose percentages of success rate range from $\approx$15--17\%. 
	\item  Concerning the exact results when the presence of noise is significant, that is with a level of 0.01,  our \textsc{Rils}--\textsc{Rols} is still performing reasonably well, having found an exact model for  34.77\% of the all problem instances. The second best is \textsc{Gplearn} which percentage of success is just 12.75\%, followed by \textsc{Dsr} which is successful for just 9.23\% of the all problem instances. 

	\item When comparing performances of the algorithms in terms of number of instances for which the algorithm's termination led to an accurate model, that is $R^2 \geq 0.999$, in case when no noise is integrated, our \textsc{Rils}-\textsc{Rols} is successful with the rate of success of 83.38\%. However, in contrast with the exact model percentages comparisons from above, there are a few methods reaching even better percentages of success than ours: \textsc{Mrgp}, and \textsc{Operon}, with 92.69\%, and 86.93\%. On the other hand, as the  exact model percentage rates of these two approaches are rather low (0\% and 16\%, respectively) these indicate the presence of a high level over-fitting  of the later two algorithms. Thus, they are able to obtain a model that is accurate enough on the training data, but most of the produced models are in essence wrong.
	
   \item  When comparing the performances of the competitors in terms of number of instances for which the  termination led to an accurate enough model, in the presence of the noise level of 0.001, our \textsc{Rils}--\textsc{Rols} method has 78.62\% of a success rate, concerning all considered instances.  The \textsc{Operon} solver only has a better percentage of success among the competitors, which is 83.08\%. However, the over-fitting is conspicuous as this algorithm is rarely able to obtain the model which matches to the exact one (which is successful for 0.31\% cases). 
   
   \item   When comparing the performances of the competitors in terms of number of instances for which the  termination led to an accurate enough model, in the presence of the high--level noise of 0.01, our \textsc{Rils}--\textsc{Rols} method is still highly accurate, delivering  62.92\% of a success rate. Interestingly, it is the best performing in terms of the percentage rate among other competitors. Just for the information, the second best approach is \textsc{Dsr}, with a 16.92\% success rate. It can be concluded that our \textsc{Rils}--\textsc{Rols} method does not pay a big price in presence of a higher noise, like the other methods do, and seem to be very sensible w.r.t. the noise increase. Our \textsc{Rils}--\textsc{Rols} method still delivers high-accuracy models for the noisy input data, but also many of them match to the exact solutions (models).
   
   \item From the above conclusions, our \textsc{Rils}-\textsc{Rols} algorithm is more robust than the other approaches, as shown by the produced results w.r.t.\ different levels of noise in the input data. Moreover, this is a new state-of-the-art algorithms when one concerns the problem of SR.
   
   
\end{itemize}


 


\begin{table}[!htb]
	\caption{Comparison without noise}\label{tab:comp_noise0}
	\centering
	\begin{tabular}{l|rrr|rrr} \hline
		& \multicolumn{3}{c|}{Exact model percentage} & \multicolumn{3}{c}{$R^2 > 0.999$ percentage}\\ \hline
		Method & Feynman & Strogatz & Total & Feynman & Strogatz & Total \\ \hline
		\textsc{Rils-Rols}&\bf{57.84}&\bf{83.57}&\bf{60.62}&82.59&90&83.38\\
		\textsc{AI-Feynman} &55.78&27.14&52.69&78.51&35.71&73.9\\
		\textsc{Gomea} &26.83&29.46&27.12&71.55&71.43&71.54\\
		\textsc{Afp-Fe} &26.98&20&26.23&59.05&28.57&55.77\\
		\textsc{Itea}&22.41&7.14&20.77&27.59&21.43&26.93\\
		\textsc{Afp}&21.12&15.18&20.48&44.83&25&42.69\\
		\textsc{Dsr}&19.72&19.64&19.71&25&14.29&23.85\\
		\textsc{Operon}&16.55&11.43&16&86.21&\bf{92.86}&86.93\\
		\textsc{gplearn}&16.27&8.93&15.48&32.76&7.14&30\\
		\textsc{Sbp-Gp}&12.72&11.61&12.6&73.71&78.57&74.23\\
		\textsc{Eplex}  &12.39&8.93&12.02&46.98&21.43&44.23\\
		\textsc{Bsr}&2.48&0.89&2.31&10.78&21.43&11.93\\
		\textsc{Feat}&0&0.89&0.1&39.66&42.86&40\\
		\textsc{Ffx}&0&0&0&0&0&0\\
		\textsc{Mrgp}&0&0&0&\bf{93.1}&89.29&\bf{92.69}\\
		\hline
	\end{tabular}
\end{table}


\begin{table}[!htb]
	\caption{Comparison with noise level 0.001}\label{tab:comp_noise0001}
	\centering
	\begin{tabular}{l|rrr|rrr} \hline
		& \multicolumn{3}{c|}{Exact model percentage} & \multicolumn{3}{c}{$R^2 > 0.999$ percentage}\\ \hline
		Method & Feynman & Strogatz & Total & Feynman & Strogatz & Total \\ \hline
		\textsc{Rils-Rols}&\bf{42.93}&\bf{35}&\bf{42.08}&80.26&65&78.62\\
		\textsc{Dsr}&16.81&18.75&17.02&26.29&14.29&25\\
		\textsc{gplearn}&15.84&8.93&15.1&24.14&7.14&22.31\\
		\textsc{Afp-Fe}&15.95&6.43&14.92&57.76&28.57&54.62\\
		\textsc{Afp}&14.87&7.14&14.04&44.83&25&42.69\\
		\textsc{Eplex} &10.02&2.68&9.23&48.71&21.43&45.77\\
		\textsc{AI-Feynman}&1.49&13.73&2.81&9.73&35.71&12.53\\
		\textsc{Itea}&1.94&4.46&2.21&26.72&21.43&26.15\\
		\textsc{Gomea}&1.62&0.89&1.54&72.84&71.43&72.69\\
		\textsc{Operon}&0.34&0&0.31&\bf{81.9}&\bf{92.86}&\bf{83.08}\\
		\textsc{Ffx}&0.11&0&0.1&19.4&14.29&18.85\\
		\textsc{Feat}&0&0.89&0.1&13.79&50&17.69\\
		\textsc{Sbp-Gp}&0&0&0&62.07&53.57&61.15\\
		\textsc{Bsr}&0&0&0&7.76&14.29&8.46\\
		\textsc{Mrgp}&0&0&0&0.86&17.86&2.69\\
		\hline
	\end{tabular}
\end{table}


\begin{table}[!htb]
	\caption{Comparison with noise level 0.01}\label{tab:comp_noise001}
	\centering
	\begin{tabular}{l|rrr|rrr} \hline
		& \multicolumn{3}{c|}{Exact model percentage} & \multicolumn{3}{c}{$R^2 > 0.999$ percentage}\\ \hline
		Method & Feynman & Strogatz & Total & Feynman & Strogatz & Total \\ \hline
		\textsc{Rils-Rols}&\bf{36.29}&\bf{22.14}&\bf{34.77}&\bf{64.91}&\bf{46.43}&\bf{62.92}\\
		\textsc{gplearn}&13.21&8.93&12.75&14.22&7.14&13.46\\
		\textsc{Dsr}&8.41&16.07&9.23&17.24&14.29&16.92\\
		\textsc{Eplex}&8.77&0&7.83&14.22&3.57&13.07\\
		\textsc{Afp}&7.11&0.89&6.44&8.19&7.14&8.08\\
		\textsc{Afp-Fe}&6.12&3.57&5.85&5.6&0&5\\
		\textsc{Itea}&0.11&0.89&0.19&1.72&10.71&2.69\\
		\textsc{AI-Feynman}&0&0.79&0.09&0&0&0\\
		\textsc{Operon}&0.09&0&0.08&0&0&0\\
		\textsc{Gomea}&0&0&0&0&0&0\\
		\textsc{Sbp-Gp}&0&0&0&0&0&0\\
		\textsc{Bsr}&0&0&0&0&0&0\\
		\textsc{Feat}&0&0&0&0&14.29&1.54\\
		\textsc{Ffx}&0&0&0&0&0&0\\
		\textsc{Mrgp}&0&0&0&0&0&0\\
		\hline
	\end{tabular}
\end{table}



\subsection{Scalability of \textsc{Rils}-\textsc{Rols} algorithm}

In this section we study scalability of our method in terms of different noise level in the input data as well as different complexity of the exact solutions (models). 

First, we divide our benchmark sets into three parts as follows. 
\begin{itemize}
	\item Small--sized instances: all instances from \textsc{Random} benchmark whose the corresponding exact models tree representations have from 3 to 7 nodes belong to this sub-set.
	\item Middle--sized instances:  all instances from \textsc{Random} benchmark whose the corresponding exact models tree representations have from 8 to 11 nodes belong to this sub-set.
	\item Large--sized instances: all instances from \textsc{Random} benchmark whose the corresponding exact models tree representations have from 12 to 15 nodes belong to this sub-set. 
\end{itemize}

The results are displayed in Figure~\ref{fig:compExact_noise_size}. They are grouped with accordance to the (three) formed sub-groups ($x$-axis) and for each of them the exact percentage rate of success of \textsc{Rils}--\textsc{Rols} algorithm is shown ($y$-axis). The following conclusions may be drawn from here:

\begin{itemize}
	\item   As expected, the success rate is the highest for the small--sized instance problems; it is sightly bellow 90\% for the clean (no-noise) data. For more noisy data, the success rate of the algorithm gets decreased. For example, on the small-sized instances with the presence of the noise of level 0.0001, and 0.01, the rate of success of \textsc{Rils}--\textsc{Rols} is still reasonably well, about 55\%~ and 35 \%, respectively. 
	\item For middle-sized instance problems, the rate of success is slightly bellow 50\%. It also decreases by increasing the level of noise in the target data; for the highest level of chosen noise (of 0.01), the success of rate is about 6\%. 
	\item For the large-sized instance problems without the presence of a noise, the rate fo success of \textsc{Rils}--\textsc{Rols} algorithm is at about 25\%, which shows that increase in the size of exact models affects the algorithm's performance, but within reasonable expectations.  
	
	\item We conclude that the size of exact models (solutions) and the increase of the noise level evenly contribute to overall performance of our \textsc{Rils}--\textsc{Rols} method. The success rate of the algorithm (i.e. the exact percentage) seems decreasing linearly  with the average growing size of the exact models, regardless of the value of the noise level.
\end{itemize}

\begin{center}
 \begin{tikzpicture}
 	\begin{axis}[
		xlabel=$size$,
		ylabel=$exact\;percentage$,
		xmin=0, xmax=4,
		ymin=0, ymax=100,
		xtick={1,2,3},
		xticklabels={small,medium,large},   % <---
		ytick={0,10,...,100}
		]
		\addplot[smooth,mark=*,blue] plot coordinates {
			(1,85)
			(2,42.5)
			(3,21.05)
		};
		\addlegendentry{No noise}
		
		\addplot[smooth,color=red,mark=x]
		plot coordinates {
			(1,51.67)
			(2,13.75)
			(3,5.26)
		};
		\addlegendentry{Noise level 0.001}
		
		\addplot[smooth,color=green,mark=o]
		plot coordinates {
			(1,31.67)
			(2,5)
			(3,4.21)
		};
		\addlegendentry{Noise level 0.01}
	\end{axis}
\end{tikzpicture}
\captionof{figure}{Exact solution percentages for varying levels of noise and formulae sizes}
\label{fig:compExact_noise_size}
\end{center}

\begin{center}
	\begin{tikzpicture}
		\begin{axis}[
			xlabel=$size$,
			ylabel=$R^2 > 0.999\;percentage$,
			xmin=0, xmax=4,
			ymin=20, ymax=100,
			xtick={1,2,3},
			xticklabels={small,medium,large},   % <---
			ytick={20,30,...,100}
			]
			\addplot[smooth,mark=*,blue] plot coordinates {
				(1,98.33)
				(2,62.5)
				(3,58.95)
			};
			\addlegendentry{No noise}
			
			\addplot[smooth,color=red,mark=x]
			plot coordinates {
				(1,93.33)
				(2,53.75)
				(3,40)
			};
			\addlegendentry{Noise level 0.001}
			
			\addplot[smooth,color=green,mark=o]
			plot coordinates {
				(1,65)
				(2,31.25)
				(3,20)
			};
			\addlegendentry{Noise level 0.01}
		\end{axis}
	\end{tikzpicture}
	\captionof{figure}{Percentages of solutions having $R^2 > 0.999$ for varying levels of noise and formulae sizes}
	\label{fig:compR2_noise_size}
\end{center}

\begin{center}
	\begin{tikzpicture}
		\begin{axis}[
			xlabel=$variable\;count$,
			ylabel=$exact\;percentage$,
			xmin=0, xmax=5,
			ymin=0, ymax=100,
			xtick={1,2,3,4},
			xticklabels={1,2,3,4},   % <---
			ytick={0,10,...,100}
			]
			\addplot[smooth,mark=*,blue] plot coordinates {
				(1,43.33)
				(2,43.33)
				(3,46.67)
				(4,46.67)
			};
			\addlegendentry{No noise}
			
			\addplot[smooth,color=red,mark=x]
			plot coordinates {
				(1,3.33)
				(2,3.33)
				(3,26.67)
				(4,26.67)
			};
			\addlegendentry{Noise level 0.001}
			
			\addplot[smooth,color=green,mark=o]
			plot coordinates {
				(1,3.33)
				(2,3.33)
				(3,3.33)
				(4,16.67)
			};
			\addlegendentry{Noise level 0.01}
		\end{axis}
	\end{tikzpicture}
	\captionof{figure}{Exact solution percentages for varying levels of noise and variable counts (only sizes ranging from 7 to 12)}
	\label{fig:compExact_noise_varcnt}
\end{center}

\begin{center}
	\begin{tikzpicture}
		\begin{axis}[
			xlabel=$variable\;count$,
			ylabel=$exact\;percentage$,
			xmin=0, xmax=5,
			ymin=0, ymax=100,
			xtick={1,2,3,4},
			xticklabels={1,2,3,4},   % <---
			ytick={0,10,...,100}
			]
			\addplot[smooth,mark=*,blue] plot coordinates {
				(1,83.33)
				(2,66.67)
				(3,73.33)
				(4,46.67)
			};
			\addlegendentry{No noise}
			
			\addplot[smooth,color=red,mark=x]
			plot coordinates {
				(1,56.67)
				(2,63.33)
				(3,66.67)
				(4,46.67)
			};
			\addlegendentry{Noise level 0.001}
			
			\addplot[smooth,color=green,mark=o]
			plot coordinates {
				(1,23.33)
				(2,43.33)
				(3,53.33)
				(4,26.67)
			};
			\addlegendentry{Noise level 0.01}
		\end{axis}
	\end{tikzpicture}
	\captionof{figure}{Percentages of solutions having $R^2 > 0.999$ for varying levels of noise and variable counts (only sizes ranging from 7 to 12)}
	\label{fig:compR2_noise_varcnt}
\end{center}


\subsection{Statistical evaluation}

\section{Conclusions and future work}\label{sec:conclusions}
  
 \newpage
 \appendix
 
 \section{Overview of RILS-ROLS python package}\label{sec:appendix-1}
  
  
  
 
\newpage
%\section*{References}
\bibliographystyle{abbrv}	
\bibliography{bib}	

	
\end{document}
